%
% The first command in your LaTeX source must be the \documentclass command.
\documentclass[sigconf]{acmart}

%
% defining the \BibTeX command - from Oren Patashnik's original BibTeX documentation.
\def\BibTeX{{\rm B\kern-.05em{\sc i\kern-.025em b}\kern-.08emT\kern-.1667em\lower.7ex\hbox{E}\kern-.125emX}}
    
% Rights management information. 
% This information is sent to you when you complete the rights form.
% These commands have SAMPLE values in them; it is your responsibility as an author to replace
% the commands and values with those provided to you when you complete the rights form.
%
% These commands are for a PROCEEDINGS abstract or paper.
\settopmatter{printacmref=false}
\copyrightyear{2019}
\acmYear{2019}
\setcopyright{acmlicensed}
\acmConference[SIGIR '19]{SIGIR '19:  ACM SIGIR Conference on Research and Development in Information Retrieval}{July 21--25, 201}{Paris, France}
\acmBooktitle{SIGIR '19:  ACM SIGIR Conference on Research and Development in Information Retrieval, July 21--25, 2019, Paris, France}
\acmPrice{}
\acmDOI{}
\acmISBN{}


% end of the preamble, start of the body of the document source.
\begin{document}


% The "title" command has an optional parameter, allowing the author to define a "short title" to be used in page headers.
\title{Convolutional Neural Networks for Identify Target Entity Types}


\begin{abstract}
Identifying the target type of entity bearing queries can help improve the overall performance of the search. In this work, we propose a deep neural network approach to solve the problem. The main benefit of this approach is that it is not necessary to provide hand-made features to the network. The main contribution of this paper is to use CNN networks, to extract different aspect of importance in target type identification problem. We show our approach outperforms the existing LTR approach by a remarkable margin.
\end{abstract}

%
% The code below is generated by the tool at http://dl.acm.org/ccs.cfm.
% Please copy and paste the code instead of the example below.
%
% \begin{CCSXML}
%	<ccs2012>
%	<concept>
%	<concept_id>10002951.10003317.10003338.10003343</concept_id>
%	<concept_desc>Information systems~Learning to rank</concept_desc>
%	<concept_significance>300</concept_significance>
%	</concept>
%	</ccs2012>
%\end{CCSXML}

%\ccsdesc[300]{Information systems~Learning to rank}

%
% Keywords. The author(s) should pick words that accurately describe the work being
% presented. Separate the keywords with commas.
\keywords{Query understanding, query types, entity search, neural network, deep learning}


% This command processes the author and affiliation and title information and builds
% the first part of the formatted document.
\maketitle



% The next two lines define the bibliography style to be used, and the bibliography file.
%\bibliographystyle{ACM-Reference-Format}
%\bibliography{bibliography}


\end{document}
