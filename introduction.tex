\section{Introduction}\label{introduction}
\par
A common practice in many search applications is to respond search queries with entities from a knowledge base. Entities, such as people, locations, or organizations, are characterized by their \emph{types}. These types are organized in a hierarchical structure in a knowledge base, referred to as a \emph{type taxonomy}. 
Identifying the entity types that are the target of search queries is a means towards understanding the underlying intent of the queries~\citep{Balog:2018:EOS}.  Furthermore, it is shown that target entity types -- either being explicitly specified as part of the query, or automatically identified -- can improve entity retrieval performance~\citep{Garigliotti:2018:IET, Balog:2011:QME:2037661.2037667, Kaptein:2013:ECS:2405838.2405908, pehcevski2010entity}. 


The primary objective of this paper is to identify target entity types of the queries automatically. Specifically, our goal is to use a type taxonomy to identify ``the most specific category of entities that are relevant to the query''~\citep{Garigliotti:2017:TTI:3077136.3080659}.
The approaches that are proposed to address the target type identification task can be  divided into two categories: (i) probabilistic models, and (ii) learning to rank (LTR) models ~\citep{Garigliotti:2017:TTI:3077136.3080659, Garigliotti:2018:IET}. 	Although LTR models provide better performance compared to the probabilistic models, a main shortcoming of these models is that they need several hand-crafted domain-specific features. Our goal is to propose a neural network model that can learn important features, without further need for feature engineering.


Building upon the previous work~\citep{Garigliotti:2017:TTI:3077136.3080659}, we develop neural network models for two probabilistic approaches: type-centric (TC) and entity-centric (EC). The type-centric approach identifies entity types using the similarity between the query and term-based representation of a type, while entity-centric approach is based on the top ranked relevant entities to the query. Developing feed-forward neural networks, we show that the performance of these models, while being higher than probabilistic models, is substantially lower than the LTR model. We, therefore, propose convolutional neural network models for  both type-centric and entity-centric approaches and further combine them into a single model to capture the most important features of the two regimes.
%that  the proposed model outperforms the probabilistic model by high margin and works slightly worse than the LTR although without any hand-crafted features.


The main contribution of this paper is to propose a novel convolutional neural network that can automatically identifies target entity of queries. Using a standard test collection, we show that our model  significantly and substantially outperforms the probabilistic baselines and is on par with the LTR approach with respect to NDCG@1. This is a remarkable outcome, given the limited amount of available training data and the fact that our model is built based on the minimal hand-engineered features present in the LTR approach; in fact many features, including the taxonomy based features are absent in our approach. The source code and result files of this work will be made publicly available.

%of This research indicates that each of these approaches captures a different aspect of query and type matching and therefore a single model which combine these models has better performance.



%A large portion of queries in web search is related to entity retrieval, each entity has a specific type which can be mapped to a hierarchical structure i.e taxonomy. General knowledge bases such as Wikipedia provide such taxonomies. In recent years, commercial search engines answer the entity bearing queries using entity cards \cite{Pound:2010:AOR:1772690.1772769}. Entity cards summarize the information need of the user and he/she can find the answer in the search page. It can enhance the user experience and reduce search session time. Accordingly, research in entity retrieval has been an important topic in the IR community in recent years.
%\par
%In order to enhance the quality of entity retrieval, several solutions e.g target type identification \cite{Garigliotti:2017:TTI:3077136.3080659}, attribute detection \cite{DargahiNobari:2018:QUV:3269206.3269245} has been proposed. In target type identification problem,  given a query, the goal is to identify the type of relevant result with respect to a given ontology. Previous research indicates that automatic target type identification can improve the precision of retrieval without complicating the user interface (i.e single search box). Target identification can enhance the performance of retrieval in both web search and product search \cite{Balog:2011:QME:2037661.2037667, Kaptein:2013:ECS:2405838.2405908, pehcevski2010entity}.
%%\par
%The previous research for type retrieval problem, can be divided into two categories i.e probabilistic and learning to rank (LTR) models. Although the LTR model has better performance in comparison with the probabilistic model but the LTR model needs several hand-crafted domain-wise features which restrict the usage of such model in cross domains.

%Here in this paper, our motivation is to provide a suitable neural network structure which can learn important features in automatic type identification problem.
%
%Our observation in this research indicates that a simple feed forward neural network can not solve the problem efficiently. Therefore in this paper, we propose a convolution neural network to capture important features of this problem. 
%
%%According to previous research, we use type centric (TC) and entity centric (EC) approaches to propose the architecture of the neural network. This research indicates that each of these approaches captures a different aspect of query and type matching and therefore a single model which combine these models has better performance.
%
%The main contribution of this paper is to use a CNN network to capture the importancy of words (i.e features) in the context of the given query, both in entity and type centric models. the proposed model outperforms the probabilistic model by high margin and works slightly worse than the LTR although without any hand-crafted features.